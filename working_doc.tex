\documentclass[letterpaper]{article}

\usepackage{url}
\usepackage[utf8]{inputenc}
\usepackage{lmodern}
\usepackage{graphicx}
\usepackage{amssymb,amsmath}
\usepackage[T1]{fontenc}
\usepackage[hmargin=1in, vmargin=1in]{geometry}
\usepackage{times}

\usepackage[shortlabels]{enumitem}
\setitemize{itemsep=2pt,topsep=2pt,parsep=0pt,partopsep=5pt}
\setenumerate{itemsep=2pt,topsep=2pt,parsep=0pt,partopsep=5pt,itemindent=13pt}


\usepackage{natbib}
\bibpunct{(}{)}{;}{a}{,}{,}

\newcommand{\ignore}[1]{}

\title{Notes on political persuasion}

\begin{document}
\maketitle

\subsection{Most applicable categories from classification}
\begin{itemize}
	\item Name Calling - Giving a bad name to "individuals, groups, nations, races, practices, beliefs and ideals he would have us condemn" to "make us form a judgment without examining the evidence"
	\item Glittering generalities - Identifying one's message with virtue by appealing to good emotions. Virtue words include "truth", "freedom", "honor", "liberty", "public service", "the American way", etc. 
	\item Testimonial - Using testimonials from trusted figures (celebrities, experts, people similar to those in audience) in support of one's message
	
	\item Plain Folks - Trying to win confidence by appearing to be like an average person (discussing "common things of life" - family, shared passtimes, experiences, etc.)
	\item Position Taking - I started off with this in the original list, but it's too common really, especially given the context.
	\item Credit Claiming - claiming responsibility for getting a specific outcome (e.g. getting a bill passed, etc.)
	\item Stereotyping - Conventional notion of an individual, group of people, country, etc. as held by a number of people.  Stereotypes can be negative or positive. Stereotypical content can be explicit or implicit (in the form of "cues" - coded language, evocative visuals, etc.)
		\begin{itemize}
			\item Slightly more nuanced than name-calling in that it's targeted at wider groups in general
			\item I think this may be somewhat subjective to tag, but it should still be doable.
		\end{itemize}
	\item Usage of slogans - brief, striking phrases
	\item Humor - in this case, taggers should probably be looking for things that are obviously jokes. This does not include wry humor or any kind of humor that's not entirely direct.
	\item Warmth - Unfortunately, after trying this for a couple of speeches, this did not end up panning out as something that was very easily reproducible. Judgements of how warm some speech is highly subjective.
	\item Patriotism - reference to patriotic appeal to appeal to the crowd ("flag waving") that involves mentions of patriotic emblems (flag, the veterans, eagle, etc.), how great the nation is, etc.
		\begin{itemize}
			\item Used very frequently - common phrasing that should be pretty easily classifiable.
			\item \emph{So, don’t let anyone ever tell you that this country isn’t great, that somehow we need to make it great again. Because this right now is the greatest country on Earth.} (Michelle Obama DNC 2016)
		\end{itemize}
	\item Repetition - repeating arguments with the same message
	\item Fear - usage of fear to motivate audience to support either a political strategy or a candidate in general.
		\begin{itemize}
			\item \emph{They’re sending people that have lots of problems, and they’re bringing those problems with us. They’re bringing drugs. They’re bringing crime. They’re rapists.}
			\item {\em Today, millions of young people are scared, worried about the future, worried what the future will hold}. (Cruz)
        	\item  {\em And yet, for so many Americans, the promise of America seems more and more distant}. (Cruz)
		\end{itemize}
	\item Emotional Anecdotes - telling stories while leveraging value-based language to appeal to the emotions of voters to convince them of a candidate's humanity.
\end{itemize}
\end{document}