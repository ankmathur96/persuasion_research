\documentclass[letterpaper]{article}

\usepackage{url}
\usepackage[utf8]{inputenc}
\usepackage{lmodern}
\usepackage{graphicx}
\usepackage{amssymb,amsmath}
\usepackage[T1]{fontenc}
\usepackage[hmargin=1in, vmargin=1in]{geometry}
\usepackage{times}

\usepackage[shortlabels]{enumitem}
\setitemize{itemsep=2pt,topsep=2pt,parsep=0pt,partopsep=5pt}
\setenumerate{itemsep=2pt,topsep=2pt,parsep=0pt,partopsep=5pt,itemindent=13pt}


\usepackage{natbib}
\bibpunct{(}{)}{;}{a}{,}{,}

\newcommand{\ignore}[1]{}

\title{Categories of Classification}

\begin{document}
\section*{Categories}
In general, when classifying, highlight the entire sentence.
\begin{itemize}
	\item Name Calling - Giving a bad name to "individuals, groups, nations, races, practices, beliefs and ideals he would have us condemn" to "make us form a judgment without examining the evidence". When looking for this, just look for cases where candidates call the candidates from opposing parties (or same party) out by name and use a negative adjective to describe them. 
	Specific guidelines: \\
	\begin{enumerate}
		\item Only include direct insults/invective and figurative examples. For example, include ``inept politicians'' and ``She chokes'', but not ``People who tell us that our way of life is being undermined by pernicious changes and dark forces beyond our control''
		\item Do not include rhetorical examples - if Bernie Sanders references what other people call him, by saying [they say] ``You're an extremist'', that's not a valid form of name calling to tag.
	\end{enumerate}
	\item Glittering generalities - Identifying one's message with virtue by appealing to good emotions. Virtue words include "truth", "freedom", "honor", "liberty", "public service", "the American way", etc. These words that are listed are good tags to look for, and, in addition to that, anything that talks in glittering language about the American promise or dream also qualifies as a glittering generality. These terms should be identified at a sentence level (if a sentence contains glittering generalities, tag the whole sentence).
	Examples: \\
	\begin{enumerate}
		\item ``the miracle of faith'' 
		\item ``We meet our commitments''
		\item ``They live on in each of us''
		\item ``We're gonna bring our jobs back''
		\item ``That is how change happens''
	\end{enumerate}
	Examples of what is \emph{not} glittering generalities:
	\begin{enumerate}
		\item ``your right to worship'' - this is just a normal reference to a right.
		\item ``this is a preposterous claim'' - this is a negative call to emotion.
	\end{enumerate}
	\item Testimonial - Using testimonials from trusted figures (celebrities, experts, people similar to those in audience) in support of one's message. Generally, one sees this in parts of the speech where the person is thanking the local candidate from the area, in an effort to tie the candidate's image to something that people from that area are used to/have heard of before. There are other instances, such as when candidates cite specific endorsements. This includes candidates talking about endorsements received as well as speakers endorsing other candidates. Some examples:
	\begin{enumerate}
		\item ``I was in Detroit recently with Bishop Jackson''
		\item ``People like pastor Darrell Scott''
		\item ``That is the Hillary I know''
		\item ``Just today in the Wall Street Journal – they have an amazing story about – amazing Wall Street Journal story today - about all of the support''
	\end{enumerate}
	\item Plain Folks - Trying to win confidence by appearing to be like an average person (discussing "common things of life" - family, shared passtimes, experiences, etc.). In general, this falls under attempts to be folksy. This is when candidates describe how they grew up in very simplistic households/middle class working families to create a relatable feel to their image. All sentences from a story that reminds you of the plain folks tag featuring the specific plain-folks style language should show up (i.e. all \emph{relevant} sentences should be tagged). This can show in many different contexts: speakers can be referencing other individuals' stories or they could be making calls to stories that reference plain folks ideals in the abstract. Here are some examples:
	\begin{enumerate}
		\item ``I know his family''
		\item ``my grandparents, they came from the heartland''
		\item ``They were Scot-Irish mostly – farmers, teachers, ranch hands, pharmacists, oil rig workers.''
		\item ``I grew up in a family that did not have a lot of money''
		\item ``It’s the conservative in Texas who said he disagreed with me on everything but he appreciated that, like him, I try to be a good dad''
		\item ``We are all brothers and sisters, and we’re all created, all everyone, created by the same God''
	\end{enumerate}
	Here are some examples of what is not plain folks: \\
	\begin{enumerate}
		\item ``Tim Kaine is as good a man, as humble and as committed a public servant as anybody that I know.''
		\item ``My grandparents knew these values weren’t reserved for one race'' - this references grandparents, but the language used is in general not referencing traditional American values in the same way as other plain folks examples.
	\end{enumerate}
	\item Credit Claiming - claiming responsibility for getting a specific outcome (e.g. getting a bill passed, etc.). The scope of this can be large - the speech can be claiming credit for a policy or it can be claiming credit for some kind of rejection of the other party's platform. The point of this is the candidate is connecting some \emph{result} to their actions. This content can be referencing a different candidate performing some action and assigning credit to that person in a positive way. Promises of action should not be included. \\Examples: \\
	\begin{enumerate}
		\item ``she fought so hard for funding to help first responders, to help the city rebuild''
		\item ``So, I picked 11 judges very much into the world of the Constitution.''
	\end{enumerate}
	Example of not credit claiming:
	\begin{enumerate}
		\item ``We’re going to have a total of 20 people, and I will pick from that group of 20 people.'' - intention to pick should be ignored.
		\item ``And they’re really, really amazing judges''
	\end{enumerate}
	\item Stereotyping - Conventional notion of an individual, group of people, country, etc. as held by a number of people.  Stereotypes can be negative or positive. Stereotypical content can be explicit or implicit (in the form of "cues" - coded language, evocative visuals, etc.) Stereotypes can be positive as well.
		\begin{itemize}
			\item Slightly more nuanced than name-calling in that it's targeted at wider groups in general
			\item I think this may be somewhat subjective to tag, but it should still be doable.
		\end{itemize}
	Consider some examples of stereotyping: \\
	\begin{enumerate}
		\item ``We have a tremendous drug problem in Cleveland''
		\item ``the African-American community is at risk of being shot''
	\end{enumerate}
	Examples of what might be misclassified as stereotyping: \\
	\begin{enumerate}
		\item ``this is a big, diverse country''
		\item ``That’s why we can attract strivers and entrepreneurs from around the globe to build new factories and create new industries here''
	\end{enumerate}
	\item Patriotism - reference to patriotic appeal to appeal to the crowd ("flag waving") that involves mentions of patriotic emblems (flag, the veterans, eagle, etc.), how great the nation is, etc. In general, this also includes references to historical greatness, shared heritage, as well as references to national improvement / restoring greatness. This can sometimes be confused with glittering generalities. Feel free to mark both labels if it is both.
		\begin{itemize}
			\item Used very frequently - common phrasing that should be pretty easily classifiable.
			\item ``So, don’t let anyone ever tell you that this country isn’t great, that somehow we need to make it great again. Because this right now is the greatest country on Earth.'' (Michelle Obama DNC 2016)
			\item ``We’re one nation, and when any one hurts, we all hurt together'' (Obama)
		\end{itemize}
	\item Repetition - repeating arguments with the same message. Select the repeated phrase, \emph{along with its repitition}. For example, if the repeated phrase is yes we can. Highlight ``yes we can. yes we can.'', not just the first ``yes we can''. Examples that are not exact repetition but are still using the same technique more loosely should be tagged as repetition.
	\item Fear - usage of fear to motivate audience to support either a political strategy or a candidate in general.
		\begin{itemize}
			\item \emph{They’re sending people that have lots of problems, and they’re bringing those problems with us. They’re bringing drugs. They’re bringing crime. They’re rapists.}
			\item {\em Today, millions of young people are scared, worried about the future, worried what the future will hold}. (Cruz)
        	\item  {\em And yet, for so many Americans, the promise of America seems more and more distant}. (Cruz)
		\end{itemize}
		Examples of what might be misclassified as fear: \\
		\begin{enumerate}
			\item ``It’s like candy being taken away from a baby.''
			\item ``Just the fanning of resentment and blame and anger and hate''
		\end{enumerate}
		When categorizing this, look for threats, discussion about current or future impacts (poverty, failure, unemployment). When annotating examples of fear, highlight the whole sentence (or full clause) - not just the fragment of the fear portion that you've identified.
	\item Emotional Anecdotes - telling stories while leveraging value-based language to appeal to the emotions of voters to convince them of a candidate's humanity. It is sometimes difficult to determine exactly how much to highlight here. Identify stories, and highlight up to 3 of the sentences most core to the story being told (or if it's a very long anecdote, break into multiple annotations). Note that emotional anecdotes do not have to be about a specific person - they can cite generic character tropes as well. \\
	Examples of emotional anecdotes: \\
	\begin{enumerate}
		\item ``Hillary’s still got the tenacity that she had as a young woman, working at the Children’s Defense Fund, going door to door to ultimately make sure kids with disabilities could get a quality education''
		\item ``She’s still seared with the memory of every American she met who lost loved ones on 9/11''
	\end{enumerate}
	Examples of what may be misclassified as emotional anecdotes: \\
	\begin{enumerate}
		\item ``But after it was all over, I asked Hillary to join my team and she was a little surprised''
		\item ``I try to be a good dad.'' - this more closely matches plain folks
	\end{enumerate}
\end{itemize}
\subsection*{Notes}
Cutting slogans.

On fear: we cannot just make this explicit fear appeals because that's most of the time not how fear is used in speech. 
\end{document}